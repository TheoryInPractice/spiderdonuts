
Different active threads:
\begin{enumerate}
  \item\label{item:sufficient-walk-matrix} In the $k$-class case, figure out conditions on walk-class-matrix $\mW$ that are \emph{sufficient} to imply the graph is deceptive (ideas listed above, includes ASFF and PWFF for example)
  \item In the $k$-class case, figure out graph properties that are sufficient to guarantee matrix property(s) determined in~\ref{item:sufficient-walk-matrix}.
  \item In the 2-class case, determine what \emph{graph} property(s) are sufficient/necessary for the Each-Class-Max matrix condition to hold.
\end{enumerate}




\begin{theorem}[Farkas]
  Fix $\mA \in \mathbb{R}^{m \times n}$ and $\vb \in \mathbb{R}^m$. Then exactly one of the following is true:
  \begin{enumerate}
    \item There exists $\vx \geq 0$ such that $\mA\vx = \vb$, XOR
    \item There exists $\vy \in \mathbb{R}^m$ such that $\vy^T\mA \geq 0$ and $\vy^T \vb < 0$.
  \end{enumerate}
\end{theorem}

This guarantees the existence of a \emph{nonnegative} solution, $\vx \geq 0$, which does not guarantee a \emph{positive} solution $\vc > 0$ the way we want.
But! The following lemma enables us to turn any nonnegative solution to Equation~\eqref{eqn:class-walk-matrix1} into a positive solution!

Let $WR(\vx)$ denote the vector $\vx$ restricted to just one row for each walk-class. (When we use the mapping $WR$ it should not matter which node/row is used to represent any particular walk-class.)

\begin{lemma}[The ``Farkas's Lemma" lemma ]\label{thm:farkaslemmalemma}
  Suppose $g(x)$ is a positive-power-series function. Construct the walk-class matrix $\mM$ for a graph $\mA$, so that each column of $\mM$ is $WR( \diag(\mA^k))$ for some $k$.
  Next, let $\vg$ be the vector corresponding to $WR(\diag(g(\mA)))$, i.e. restricted to the distinct walk-classes.
  Let $\vx$ be any nonnegative solution to $\mM \vx = \gamma\ve - \vg$.
  Then there exists a positive-power-series function $h(x)$ different from $g(x)$ in a finite number of terms such that
  $\mM\vc = \gamma\ve - \vh$, and $\vc > 0 $, where $\vc \equiv \vx$ except on the entries where $\vx$ was zero.
\end{lemma}

% Note that the positive solution $\vc$ enables the

\begin{proof}
  Let $\vx$ be a nonnegative solution to $\mM\vx = \gamma\ve - \vg$, and
  let $\mathcal{J}$ be the index set containing the entries of $\vx$ that are zero. Note that for each $j \in \mathcal{J}$ there exists some power $p_j$ such that $\mM(:,j) = WR(\diag(\mA^{p_j}))$. Let $\mathcal{P}$ be the set of such powers, $p_j$.

  Define a new function $h(x)$ by adding to $g(x)$ a small amount at each of the terms in $\mathcal{J}$. Formally, fix any $\tau >0$ and set $h(x) = g(x) + \sum_{j \in \mathcal{J}} \tau x^{p_j}$. Thus, if we write $h(x) = \sum_{k=0}^{\infty} h_k x^k$, then we have
  \[
  h(x) = \left(\sum_{j \notin \mathcal{J}} g_j x^{p_j}\right) + \left( \sum_{j \in \mathcal{J}} (g_j + \tau) x^{p_j}\right),
  \]
  which proves that $h$ has a power-series that differs from the power-series of $g$ in a finite number of terms.

  Next, note that $\vh = \vg + \tau \cdot \mM \ve$. This is because
  \begin{align}
    \vh &= \sum_{k=0}^{\infty} h_k WR(\diag(\mA^k)) \\
        &= \sum_{j \in \mathcal{J} } h_{p_j} WR(\diag(\mA^{p_j})) +  \sum_{k \notin \mathcal{P} } h_k WR(\diag(\mA^k)) \\
        &= \sum_{j \in \mathcal{J} } (g_{p_j} + \tau) WR(\diag(\mA^{p_j})) +  \sum_{k \notin \mathcal{P} } h_k WR(\diag(\mA^k)) \\
        &= \tau \sum_{j \in \mathcal{J} } WR(\diag(\mA^{p_j})) + \sum_{k \notin \mathcal{P} } h_k WR(\diag(\mA^k)) \\
        &= \tau \mM\ve_{\mathcal{J}} +\vg
  \end{align}
  as desired.
  Plugging this into the expression $\mM \vx = \gamma\ve - \vg$, we get $\mM \vx = \gamma\ve - (\vh - \tau \mM\ve_{\mathcal{J}})$, and rearranging yields the equation
  $\mM(\vx+\tau \ve_{\mathcal{J}}) = \gamma\ve - \vh$.
  Thus, $\vc = \vx + \tau\ve_{\mathcal{J}}$ is an entirely positive solution, since we defined $\mathcal{J}$ to be the set of indices of $\vx$ where $\vx$ was zero.

\end{proof}

This lemma allows us to use Farkas's Lemma to construct positive solutions for the purposes of our problem.
