% sec-graph-tensor-product.tex

% An infinite family of deceptive graphs via the graph tensor product
% \Box give cartesian product, \otimes gives the tensor product

Given two graphs $G$ and $H$, there are a number of different ``products" defined that can produce a third, new graph that somehow combines the two. In this section we consider in particular the tensor product of two graphs, denoted $G \times H$ or by $G \otimes H$. We will use $G \otimes H$ because of a connection with the tensor product of the graphs' adjacency matrices $\mA_{G} \otimes \mA_{H} = \mA_{G \otimes H}$.

The graph $G \otimes H$ is defined as follows. Let $G = (V,E)$ and $H = (U,F)$ be graphs. Then $G\otimes H$ has one node for each tuple $(v,u)$ with $v \in V$ and $u \in U$ (so, $|V|\cdot|U|$ total nodes); then, nodes $(v_1, u_1)$ and $(v_2, u_2)$ are connected by an edge in $G \otimes H$ iff an edge connects $v_1, v_2$ in $G$ \emph{and} an edge connects $u_1, u_2$ in $H$.

\subsection{Constructing a deceptive function on a tensor graph}\label{sec:tensor-construction}

Suppose we have a deceptive graph $G$ and a walk-regular graph $H$. We will construct a function $f(x) = \sum_{k=0}^{\infty} c_k x^k$ with $c_k >0$ such that $f( \mA_{G \otimes H})$ has constant diagonal. Then, if we can show the graph $G \otimes H$ is connected (which it will be under certain conditions on $H$) we will have demonstrated it is deceptive.

To construct $f(\mA_{G \otimes H})$ to have constant diagonal, consider its diagonal entry $(\ve_i \otimes \ve_j)^T f( \mA_{G \otimes H}) (\ve_i \otimes \ve_j)$. Recall that $\mA_{G \otimes H} = \mA_{G} \otimes \mA_H$. Next we expand the power series definition of $f$ and show how we can choose $c_k$ so that this diagonal entry is a constant independent of $i, j$.
\begin{align}
  f(\mA_{G \otimes H})_{\textrm{diag}} &= (\ve_i \otimes \ve_j)^T f( \mA_{G \otimes H}) (\ve_i \otimes \ve_j) \\
  &= (\ve_i \otimes \ve_j)^T f( \mA_G \otimes \mA_H) (\ve_i \otimes \ve_j) \\
  &= (\ve_i \otimes \ve_j)^T \left( \sum_{k=0}^{\infty} c_k (\mA_{G} \otimes \mA_H)^k \right) (\ve_i \otimes \ve_j) \\
  &= \sum_{k=0}^{\infty} c_k (\ve_i^T(\mA_{G}^k)\ve_i) \otimes (\ve_j^T(\mA_H^k)\ve_j ) \\
  &= \sum_{k=0}^{\infty} c_k (\mA_{G}^k)_{ii} (\mA_H^k)_{jj}
\end{align}
By assumption, the graph $H$ is walk-regular, and so for a fixed $k$ we know $(\mA_H^k)_{jj}$ is a constant for all $j$. This let's us choose $c_k$ as follows.
Again by assumption we know that $G$ is deceptive, and so there exists a positive sequence $d_k > 0$ such that $\sum_{k=0}^{\infty} d_k (\mA_G^k)$ has constant diagonal.
We set
\[
c_k := \begin{cases}
d_k & \textrm{ if } (\mA_H^k)_{jj} = 0 \\
d_k/(\mA_H^k)_{jj} & \textrm{ otherwise.}
\end{cases}
\]
Note that it doesn't matter which value we choose for $j$ since the matrix $\mA_H^k$ has constant diagonal. Returning to our evaluation of $f(\mA_{G \otimes H})$ above, by plugging in our definition of $c_k$ we can simplify to get
\begin{align}
  f(\mA_{G\otimes H})_{\textrm{diag}} &= \sum_{k=0}^{\infty} d_k (\mA_{G}^k)_{ii},
\end{align}
and since $\sum{k=0}^{\infty}d_k \mA_G^k$ has constant diagonal (by construction of $d_k$), we have that $f(\mA_{G\otimes H})_{\textrm{diag}}$ is constant. This completes a proof of the following proposition.

\begin{proposition}
  Given a deceptive graph $G$ and a walk-regular graph $H$, we can construct a positive sequence $c_k > 0$ such that the function $f(x) = \sum_{k=0}^{\infty} c_k x^k$ has constant diagonal on the graph $G \otimes H$.
\end{proposition}

In order for us to conclude that $G \otimes H$ is \emph{deceptive} we have to show that $G \otimes H$ is connected. If it is connected, then by definition of deceptive, $G \otimes H$ is also deceptive.

\begin{corollary}
  Given a deceptive graph $G$ and a walk-regular graph $H$, if $G \otimes H$ is connected then it is also deceptive.
\end{corollary}

So, what conditions on $G$ and $H$ can guarantee that $G \otimes H$ is connected? Keep in mind we are assuming that $G$ is connected (because we assume it is deceptive, which requires connectivity). A theorem of Weichsel~(\cite{weichsel1962kronecker}, Theorem 1) states that any tensor graph $G_1 \otimes G_2$ is connected iff both $G_1$ and $G_2$ are connected and at least one of them contains a cycle of odd length.
Therefore, for any connected, walk-regular graph $H$ that contains an odd cycle, we have that $G\otimes H$ is deceptive. Note that the cycle graphs of odd length, $C_{(2k+1)}$, are an infinite family of such connected walk-regular odd-cycle graphs, and so $G \otimes C_{(2k+1)}$ gives a distinct deceptive graph for each $k \geq 1$.

\begin{corollary}
  There exist infinite families of deceptive graphs.
\end{corollary}


\section{Cartesian product graphs}\label{sec:cartesian-graph}

We suspect that a similar result will hold for the Cartesian product of two graphs $G,H$, denoted $G \Box H$. The definition of $G \Box H$ is very similar to that of $G \otimes H$ with one important difference.
The construction of the nodeset is the same: $G \Box H$ has nodes $(u, v)$ for every node $u$ of $G$ and node $v$ of $H$. Given two nodes in $G \Box H$,  $(u_1, v_1)$ and $(u_2, v_2)$, they are connected by an edge iff (1) $v_1 = v_2$ and edge $\{u_1, u_2\}$ exists in $G$ OR (2) $u_1 = u_2$ and edge $\{v_1,v_2\}$ exists in $H$. (In contrast, $G \otimes H$ would include the edge iff \emph{both} smaller edges existed.)

The adjacency matrix of $G \Box H$ is $\mA_{G \Box H} = \mA_G \otimes \mI + \mI + \mI \otimes \mA_H$. This structure is a bit more complicated than that of $\mA_{G \otimes H}$, so the proof that tensor product graphs can be deceptive does not hold for these Cartesian product graphs.

As a cute note, observe that, since $\exp(\mA + \mB) = \exp(\mA)\exp(\mB)$ for any matrices $\mA$ and $\mB$ that commute with each other, we have that $\exp( \mA_{G \Box H}) = \exp(\mA_G \otimes \mI ) \exp(\mI \otimes \mA_{H}) = \exp(\mA_G)\otimes \exp(\mA_H)$. The cute part is that if $\exp(\mA_G)$ and $\exp(\mA_H)$ have constant diagonal, then so does $\exp( \mA_{G \Box H})$ -- this means that if $\exp$ was deceptive on two different graphs $G$ and $H$, then $\exp$ would be deceptive on any $\Box$ product $G \Box H$, $G \Box (G \Box H)$, etc. (Estrada and de la Pena proved that $\exp$ cannot be deceptive on any graph, so this point is moot.)



Some useful lemmas for our exploration of when a Cartesian product graph is deceptive:

\begin{lemma}
  Showe lemma: let $\mX$ and $\mY$ be arbitrary square, symmetric matrices. Then for any positive integer power $p$,
  \[
  (\mX \otimes \mI + \mI \otimes \mY)^p = \sum_{t=0}^p \binom{p}{t} \mX^{p-t} \otimes \mY^t
  \]
\end{lemma}

\begin{lemma}
  Showe lemma: let $\mX$ and $\mY$ be arbitrary square matrices. Let $m$ be a positive integer, and $c_k$ be any coefficients.
  \begin{align}
  \sum_{k=0}^m c_k (\mX \otimes \mI + \mI \otimes \mY)^k &=
    \sum_{k=0}^m c_k \left(\sum_{s=0}^k \binom{k}{s} \mX^{k-s}\otimes \mY^s \right) \\
    &= \sum_{s=0}^m \left( \sum_{k=s}^m c_k \binom{k}{s} \mX^{k-s} \right)\otimes \mY^s
  \end{align}
\end{lemma}


Outline of progress:

\begin{itemize}
  \item We want to use our lemma that to prove deceptiveness it suffices to construct a polynomial $p(x) = \sum_{k=0}^m c_k x^k $  (rather than an infinite series) with positive coefficients such that $p(\mA)$ has constant-diagonal.
  \item Using our lemmas about graph Cartesian products, we have $p( \mA_{H \Box G}) = \sum_{s=0}^m \left( \sum_{k=s}^m c_k \binom{k}{s} \mA_H^{k-s} \right)\otimes \mA_G^s$.
  \item By assumption, $\mA_H$ is walk-regular, and so the diagonal $\diag(\mA_H^{k-s})$ are constant for each fixed $k,s$.
  \item Hence, the diagonal entries $(\ve_i \times \ve_j)^T p(\mA_{H\Box G})(\ve_i \times \ve_j) = \sum_{s=0}^m \left( \sum_{k=s}^m c_k \binom{k}{s} (\mA_H^{k-s})_{ii} \right)\otimes (\mA_G^s)_{jj}$ depend only on $j$, not on $i$.
  \item Set $h(k,s) := (\mA_H^{k-s})_{ii}$ and note the choice of $i$ does not matter. Then we've reduced the problem to choosing values $c_k$ such that $ \sum_{s=0}^m \left( \sum_{k=s}^m c_k \binom{k}{s} h(k,s) \right)\otimes (\mA_G^s)_{jj}$ is contsant for different $j$.
  \item Setting $\vv_s = \diag( \mA_G^s )$, we're really just looking for $c_k$ that make $\sum_{s=0}^m \left( \sum_{k=s}^m c_k \binom{k}{s} h(k,s) \right)\otimes \vv_s = \ve$.
  \item Set $\mV = [ \vv_0, \vv_1, \cdots, \vv_m ]$ and define the upper-triangular matrix $\mF$ as follows; for any $k \geq s$, define $\mF(s,k) = \binom{k}{s} h(k,s)$. Then we are looking for a positive solution $\vc$ to the equation $\mV\mF\vc = \ve$.
  \item We know that the existence of a positive solution $\vx$ to $\mV\vx = \ve$ implies that $G$ is deceptive. We \emph{think} that the converse is true, but have not proved it yet.
  \item Supposing that a positive solution $\vx$ exists to $\mV \vx = \ve$, then proving deceptiveness could be accomplished by constructing a positive solution $\vx$ to $\vx = \mF\vc$. We know that $\mF$ is invertible (because it is upper triangular), and $\mF$ is nonnegative. It is unclear if positive $\vx$ necessarily means $\vc = \mF\inv \vx$ is positive. (For general positive upper-triangular $\mF$ it is not the case, but we have additional structure.)
  \item Expanding the matrix equation $\mF\vc = \vx$ yields $c_m = x_m$ and $c_{j-1} = x_{j-1} - (\sum_{i=j}^m c_i F(j-1,i))$.
\end{itemize}
