\subsection{Structure inherent to graph matrices}\label{sec:fundamentals:graphstructure}

There are a lot of nice properties that matrices and their eigenvalues/vectors can have if the matrices are ``well-behaved" in a sense. It turns out, many graph matrices enjoy a lot of these nice properties! Here we discuss some of the most useful of these properties. What do we mean by well-behaved? Think of polynomials -- the polynomial p(x) = x^2-1 is very well-behaved in the sense that all of its rooots are integers; in contrast, a polynomial like p(x) = x^2 + 2 has roots that are complex numbers with irrational components ($\sqrt{2}$). This is not a perfect analogy but it gives the intution that a ``well-behaved" object is easier to work with. Next we cover two examples of what we mean for matrices to be well-behaved in our context.

\begin{proposition}
  Let $G$ be a standard graph with incidence matrix $\mB$ and Laplacian matrix $\mL$. Then $\mL = \mB^T\mB$.
\end{proposition}
