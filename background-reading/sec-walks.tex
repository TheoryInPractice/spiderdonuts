\subsection{Graph walks, graph matrices}\label{sec:fundamentals:walks}


Here we look at what I call the Fundamental Lemma of Network Analysis because of how often it is used.
\begin{lemma}[Fundamental Lemma of Network Analysis]\label{thm:fundamental-walk}
Let the graph $G$ have adjacency matrix $\mA$. Then the number of length $k$ walks from node $i$ to node $j$ is equal to $(\mA^k)_{ij}$.
\end{lemma}

This lemma is so often used because it gives us a way to measure connectivity between different nodes -- it gives us a way to measure, for a fixed node $s$, what nodes are most important to, or similar to, the node $s$?
Consider the polynomial $p(x) = \sum_{k=1}^N x^k$. Then the vector
\[
\vf =  p(\mA)\ve_s = \sum_{k=1}^N (\mA^k)\ve_s
\]
tells us for each node $j$ in the graph, how many total walks are there from node $s$ to node $j$ of length $\leq N$ ?
Large entries in the vector $\vf$ end up being good predictors for nodes that are important to node $s$.






\subsection{Exercises}

\begin{enumerate}[label=\ref{sec:fundamentals}.\arabic*]
\item Prove Lemma~\ref{thm:fundamental-walk}. Hint: begin with $\mA^1$ as the base case, and proceed by induction. For the case $(\mA^k)_{ij}$, think about the question ``how do you make a walk from node $i$ to node $j$ that is of length $k$?".

\end{enumerate}
